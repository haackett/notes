\documentclass[a4paper,10pt]{article}
\usepackage[utf8]{inputenc}
\newcommand{\myparagraph}[1]{\paragraph{#1}\mbox{}\\}
\usepackage[a4paper,margin=3.5cm]{geometry} %Sets the page geometry
\usepackage{url}
\usepackage{dirtytalk}
\usepackage{graphicx} % Package for \includegraphics
\usepackage{wrapfig} % Figure wrapping
\usepackage[T1]{fontenc} % Output font encoding for international characters
\setlength{\parskip}{1em} % Set space when paragraphs are used
\usepackage{amssymb}
\usepackage{amsmath}
\usepackage{tcolorbox}
\usepackage{mathtools}
\usepackage{tikz-cd}

% Lets you use \blankpage to make a blank page
\newcommand{\blankpage}{
\newpage
\thispagestyle{empty}
\mbox{}
\newpage
}

% Self Explanatory
\newtheorem{theorem}{Theorem}[section]
\newtheorem{definition}{Definition}[section]
\newtheorem{corollary}{Corollary}[theorem]
\newtheorem{lemma}[theorem]{Lemma}
\newenvironment{proof}{\paragraph{Proof}}{\hfill$\square$}

% Other
\DeclarePairedDelimiter\floor{\lfloor}{\rfloor} %Floor function
\setcounter{section}{-1}

\begin{document}
\title{Linear Algebra II Notes \\
\Large{Factorization of Linear Maps}}
\author{Aidan Hackett}
\maketitle

\section{Motivation}

\subsection{The Big Picture}
Math:

$$hard \longrightarrow simple$$
$$hard \longrightarrow linear$$

\subsection{}
\begin{equation}
	\left\{ \begin{aligned} 
		a_{11}x_{1} + a_{12}x_{2} + a_{13}x_{3} = b_{1} \\
		a_{21}x_{1} + a_{22}x_{2} + a_{23}x_{3} = b_{2} \\
		a_{31}x_{1} + a_{32}x_{2} + a_{33}x_{3} = b_{3} \\
	\end{aligned} \right.
\end{equation}



Suppose, $A=PDP^{-1}$, where $D$ is diagonal.

$$A\vec{x}=(A=PDP^{-1})\vec{x} = \vec{b}.$$
$$DP^{-1}\vec{x} = P^{-1}\vec{b.}$$

Let $$\vec{c} := P^{-1}\vec{b}.$$

$$DP^{-1}\vec{x} = \vec{c}.$$

... we've made the "difficult" simple.

\subsection{Discrete Time Evolution Example}

Suppose $\vec{x}_t \in \mathbb{R}^m$ is the state of a system at time A, where $t\in\mathbb{Z}_{\geq 0}$ and the system evolves by:

$$ \vec{x}_{t+1} = A\vec{x}_t$$

where $A : \mathbb{R}^n \rightarrow \mathbb{R}^n$, A linear.

Then

$$\vec{x}_1 = A\vec{x}_0.$$
$$\vdots$$
$$\vec{x}_{n+1} = A\vec{x}_n = A^{n+1}\vec{x}_0.$$

Suppose $A=PDP^{-1}$. Then

$$ A^2 = (PDP^{-1})^2 = (PDP^{-1})(PDP^{-1}) = PD^2P^{-1}$$
$$\implies A^{n+1} = PD^{n+1}P^{-1}.$$

Let $\vec{c} := P^{-1}\vec{x}$. Then

$$PD^{n+1}\vec{c} = \vec{x}_{n+1}$$

\subsection{Factoring Maps}
\subsection{Some Goals for Factoring Maps}
For a linear map

$$A : V \rightarrow W$$

here are the potential factorizations of $A$ in order of "niceness":

1. Spectral Theorem
2. Diagonalization
3. Jordan Canonical Form
4. Singular Value Decomposition

\section{Vector Spaces}
\subsection{Note}
\subsection{Definition: Vector Space}
\begin{definition}
	A vector space over a field $F$ is a set $V$ with an addition operation
and a scalar multiplication operation such that $\forall \vec{x},\vec{y} \in V,
a,b\in F:$
\begin{enumerate}
	\item $\vec{x} + \vec{y} = \vec{y} + \vec{x}$
	\item $(\vec{x} + \vec{y}) + \vec{z} = \vec{x} + (\vec{y} + \vec{z})$
	\item $\exists \vec{0} \in V : \vec{0} + \vec{x} = \vec{x}$
	\item $\exists \vec{-x} \in V : \vec{x} + \vec{-x} = \vec{0}$
	\item $a(\vec{x} + \vec{y}) = a\vec{x} + a\vec{y}$
	\item $(a+b)\vec{x} = a\vec{x} + b\vec{x}$
	\item $1 \cdot \vec{x} = \vec{x}$
	\item $a(b\vec{x}) = (ab)\vec{x}$
\end{enumerate}

\end{definition} 
\subsection{Todd-given Examples}
\begin{enumerate}
	\item $\mathbb{R}^n, \forall n \in \mathbb{N}.$ 
	\item $\mathbb{C}^n, \forall n \in \mathbb{N}.$ 
	\item $M_{nm}(\mathbb{R})$
	\item Let $F$ be a field and $S$ a set. Then define $F^S$ as the set of all functions $f: S \rightarrow F$. $F^S$ forms a vector space.
\end{enumerate}







\section{Diagonalization}
\subsection{Motivation}
\begin{enumerate}
\item[(a)] Suppose $\vec{x}_t \in \mathbb{R}^n$ represents the state of some system at time $t \in \mathbb{Z}$. Further, suppose $A: \mathbb{R}^n \to \mathbb{R}^n$ is linear with $\vec{x}_{t+1} = A\vec{x}_t$.

Given iniital state $\vec{x}_0 \in \mathbb{R}^n$:
$$\vec{x}_1 = A \vec{x}_0$$
$$\vec{x}_2 = A(A \vec{x}_0) = A^2 \vec{x}_0$$
$$\vdots$$
$$\vec{x}_{k+1} = A\vec{x}_k = \ldots = A^{k+1}\vec{x}_0$$

Suppose:
\begin{center}
\begin{tikzcd}[%
        ,every arrow/.append style={maps to}
        ,every label/.append style={font=\normalsize}
		,sep=huge
        ]
\mathbb{R}^n \arrow[r, "A"] \arrow[d, "P"'] & \mathbb{R}^n                    \\
\mathbb{R}^n \arrow[r, "D"']                & \mathbb{R}^n \arrow[u, "P^{-1}"']
\end{tikzcd}
	
\end{center}

$$A^2 = (PDP^{-1})^2$$
$$=PDP^{-1} PDP^{-1}$$
$$=PD^2P^{-1}$$

In general, $A^{k+1} = P^{k+1}P^{-1}$.

If $D=$diag$(d_1,...,d_n)$, then 
$$D^{k+1} = \text{diag}(d_1^{k+1}, \ldots, d_n^{k+1}).$$

\item[(b)] Let $T_{x=0}: \mathbb{R}^2 \to \mathbb{R}^2$ be the map that reflects points across the $y$-axis.

$$T\left(
\begin{bmatrix}
	x \\
	y \\
\end{bmatrix}
\right)
=
\begin{bmatrix}
	-x \\
	y \\
\end{bmatrix}
$$
If $\mathcal{S} = 	\{ e_1, e_2 \}$ be the std. basis for $\mathbb{R}^2$, then

$$
[T]_\mathcal{S}^\mathcal{S} = [[T(e_1)]_\mathcal{S}, [T(e_2)]_\mathcal{S}]
=
\begin{bmatrix}
	-1 & 0 \\
	0 & 1 \\
\end{bmatrix}
$$
Now, let $T_{y=2x}: \mathbb{R}^2 \to \mathbb{R}^2$ be the map that reflects points across $y=2x$.

Let's pick 
$$\begin{bmatrix} 
1 \\
2 \\
\end{bmatrix} 
\text{ and }
\begin{bmatrix} 
-2 \\
1 \\
\end{bmatrix} 
$$
as our basis vectors. Call this basis $\mathcal{B}$.

Note:
$$
T\left(
\begin{bmatrix} 
1 \\
2 \\
\end{bmatrix} 
\right)
=
\begin{bmatrix} 
1 \\
2 \\
\end{bmatrix} 
\text{ and }
T\left(
\begin{bmatrix} 
-2 \\
1 \\
\end{bmatrix} 
\right)
=
\begin{bmatrix} 
2 \\
-1 \\
\end{bmatrix} 
$$
Now, 
$$
[T_{y=2x}]_\mathcal{B}^\mathcal{B} = 
\begin{bmatrix}
	1 & 0 \\
	0 & -1 \\
\end{bmatrix}
$$
Then
$$
[T(\vec{x})]_\mathcal{B} = [T_{y=2x}]_\mathcal{B}^\mathcal{B}[\vec{x}]_\mathcal{B}
$$
Problem: Given $\vec{x} \in \mathbb{R}^2$, how do we find $[\vec{x}]_\mathcal{B}$? Need a change-of-basis matrix:

$$
Q:= [I]_\mathcal{B}^\mathcal{S} =
\begin{bmatrix}
	1 & -2 \\
	2 & 1 \\
\end{bmatrix}
$$
Then
$$
Q^{-1} = [I]_\mathbb{S}^\mathbb{B} =
\frac{1}{5}
\begin{bmatrix}
	1 & 2 \\
	-2 & 1 \\
\end{bmatrix}
.
$$

So $[T]_\mathcal{B}^\mathcal{B} = Q^{-1}[T]_\mathcal{S}^\mathcal{S}Q$. Solve for A:
$$A=
\begin{bmatrix}
	1 & -2 \\
	2 & 1 \\
\end{bmatrix}
\begin{bmatrix}
	1 & 0 \\
	0 & -1 \\
\end{bmatrix}
\frac{1}{5}
\begin{bmatrix}
	1 & -2 \\
	2 & 1 \\
\end{bmatrix}
=
\frac{1}{5}
\begin{bmatrix}
	-3 & 4 \\
	4 & 3 \\
\end{bmatrix}
$$

So 
$$
T_{y=2z}\left(
\begin{bmatrix}
	x \\
	y \\
\end{bmatrix}
\right)
=
\frac{1}{5}
\begin{bmatrix}
	-3x+4y \\
	 4x+3y  \\
\end{bmatrix}
$$

\begin{center}
\begin{tikzcd}[%
        ,every arrow/.append style={maps to}
        ,every label/.append style={font=\normalsize}
		,sep=huge
        ]
\mathbb{R}^n \arrow[d, "Q^{-1}"'] \arrow[r, "T_{y=2x}"]  & \mathbb{R}^n                 \\
\mathbb{R}^n \arrow[r, "{[T]_\mathcal{B}^\mathcal{B}}"'] & \mathbb{R}^n \arrow[u, "Q"']
\end{tikzcd}
	
\end{center}

\end{enumerate}		

\subsection{}
\begin{definition}
A linear map $T: V \to V$, $V$ f.d, is called \textbf{diagonalizable} if there 
exists a basis $\mathcal{B}$ for $V$ such that $[T]_\mathcal{B}^\mathcal{B}$ is a diagonal matrix.
\end{definition}

\subsection{Example}
If $\mathcal{B} = \{ \vec{v}_1, \ldots, \vec{v}_n \}$ is a basis for $V$ such 
that $T: V \to V$ has $[T]_\mathcal{B}^\mathcal{B}$ diagonal with
$$[T]_\mathcal{B}^\mathcal{B} = D = \text{diag}(d_1, \ldots, d_n)$$
then,
$$[T(\vec{v}_i)]_\mathcal{B} = [T]_\mathcal{B}^\mathcal{B} [\vec{v}_i]\mathcal{B}$$
$$=
\begin{bmatrix} 
d_1	&	& 0\\
	&	\ddots & \\
0	&	&	d_n\\
\end{bmatrix} 
\begin{bmatrix} 
0 \\
\vdots \\
1\\
\vdots\\
0
\end{bmatrix} 
=
d_i[\vec{v}_i]_\mathcal{B}
$$


\end{document}

